\documentclass[11pt]{article}

\usepackage{sectsty}
\usepackage{graphicx}
\usepackage{listings}
\usepackage{amsmath}
\usepackage[normalem]{ulem}

% Margins
\topmargin=-0.45in
\evensidemargin=0in
\oddsidemargin=0in
\textwidth=6.5in
\textheight=9.0in
\headsep=0.25in

\title{ Tarea 06 - EXPLICIT REFS}
\author{ Enrique Giottonini, Miguel Navarro }
\date{Octubre 26, 2022}

\begin{document}
\maketitle	


\section*{Exercise 4.8 [$\star$]}
Show exactly where in our implementation of the store these operations take linear time rather than constant time.\\
\begin{itemize}


\item En \textit{newref-exp} se utiliza \textit{append} el cual tiene un tiempo de ejecución dependiente de la cantidad de todos los argumentos en la lista (exceptuando el último) por lo que new-ref tiene tiempo de ejecución lineal. Esto siempre sucede.\\
\item En \textit{deref-exp} se utiliza \textit{list-ref}, el cual tiene un tiempo de ejecución lineal por lo que \textit{deref-exp} tiene un tiempo de ejecución lineal. Esto sucede siempre y cuando el \textit{store} no es vacío y la referencia es válida.\\
\item En \textit{setref-exp} se utiliza \textit{list-set} el cual tiene un tiempo de ejecución lineal por lo que \textit{setref-exp} tiene tiempo de ejecución lineal. Esto sucede siempre y cuando el \textit{store} no es vacío y la referencia es válida. \\
\end{itemize}

\section*{Exercise 4.9 [$\star$]}
Implement the store in constant time by representing it as a Scheme
vector. What is lost by using this representation? \\ \\
El código se vuelve más complicado de entender, además por el uso de la estructura para el vector es más complicado manejar múltiples \textit{stores}.
\newpage
\section*{Exercise 4.10 [$\star$]}
Implement the begin expression as specified in exercise 4.4.
\subsection*{Sintáxis Concreta y Abstracta}
\begin{align*}
\text{Expression}	&::= \text{begin Expression \{; Expression\}*}
\end{align*}

\begin{center}
(begin-exp exp exps)
\end{center}

\subsection*{Semántica}

\begin{lstlisting}[mathescape]
(value-of (begin-exp exp exps) env $\sigma$) =
(if (null? exps)
	(value-of exp env $\sigma$)
	(value-of (begin-exp (first exps) (rest exps)) env $\sigma_1$))
\end{lstlisting}
Donde $\sigma_1$ es el \textit{store} resultante de:
\begin{lstlisting}[mathescape]
(value-of exp env $\sigma$) 
\end{lstlisting}
\section*{Exercise 4.11 [$\star$]}
Implement list from exercise 4.5.
\subsection*{Sintáxis Concreta y Abstracta}
\begin{align*}
\text{Expression}	&::= \text{list (Expression*)}
\end{align*}

\begin{center}
(list exps)
\end{center}

\subsection*{Semántica}

\begin{lstlisting}[mathescape]
(value-of (list-exp exps) env $\sigma$) =
(if (null? exps)
	((null-val), $\sigma$)
	((pair-val (cons $val_1$ $val_2$)), $\sigma_2$)))
\end{lstlisting}
Donde:
\begin{lstlisting}[mathescape]
(value-of (first exps) env $\sigma$) = ($val_1$, $\sigma _1$)
(value-of (list-exp rest exps) env $\sigma_1$) = ($val_2$, $\sigma _2$)
\end{lstlisting}

\section*{Exercise 4.12 [$\star \star \star$]}
Our understanding of the store, as expressed in this interpreter,
depends on the meaning of effects in Scheme. In particular, it depends on us knowing
when these effects take place in a Scheme program. We can avoid this dependency by
writing an interpreter that more closely mimics the specification. In this interpreter,
value-of would return both a value and a store, just as in the specification. A fragment of this interpreter appears in figure 4.6. We call this a store-passing interpreter.
Extend this interpreter to cover all of the language EXPLICIT-REFS.
Every procedure that might modify the store returns not just its usual value but also a
new store. These are packaged in a data type called answer. Complete this definition
of value-of. \\ \\

\noindent Desde un principio la implementamos de esta forma, por tanto las pruebas y la implementación están explicit-refs.rkt, y test-explicit-refs.rkt.

\section*{Exercise 4.13 [$\star \star \star$]}
Extend the interpreter of the preceding exercise to have procedures of multiple arguments.
\subsection*{Sintáxis Concreta y Abstracta}
\begin{align*}
\text{Expression} &::= \text{proc \{Expression\}}^+\text{ Expression} \\
\text{Expression} &::= \text{(Expression \{Expression\}}^+\text{)}
\end{align*}

\begin{center}
(proc-exp params body) \\
(call-exp fun args)
\end{center}

\subsection*{Semántica}

\begin{lstlisting}[mathescape]
(value-of (proc-exp params body) env $\sigma$) = ((proc-val f), $\sigma$))

where:

f = (procedure $p_1$ p' env)
params = $(p_1 \, \, p_2 \, \,  p_3 \, \, ...\, \,  p_n)$
(proc-exp params body) = (proc-exp $p_1$ (proc-exp $p_2$ (... (proc-exp $p_n$ body))))
p' = (proc-exp $p_2$ (proc-exp $p_3$ (... (proc-exp $p_n$ body)))


(value-of (call-exp fun args) env s) = $(val_n, s_n)$

where:
args = $(a_1 \,\,a_2\,\, ... \,\,a_n)$
(call-exp fun args) = (call-exp (call-exp (... (call-exp fun $a_1$) ...) $a_{n-1}$) $a_n$)))) 




--------------------------------------------------------------------
Usando las definiciones que ya teniamos anteriormente, sabemos que: 

(value-of (call-exp rator rand) env $\sigma$)
 = (let ([proc (expval->proc $val_1$)]
         [arg $val_2$])
     (apply-procedure proc arg $\sigma_2$)) 
     
where:
$(val_1, \sigma_1)$ = (value-of rator env $\sigma$)
$(val_2, \sigma_2)$ = (value-of rand env $\sigma_1$)

(define (apply-procedure proc val $\sigma$)
  (unless (procedure? proc)
    (error 'value-of "no es un procedimiento: ~e" proc))
  (let ([var (procedure-var proc)]
        [body (procedure-body proc)]
        [saved-env (procedure-saved-env proc)])
    (value-of body (extend-env var val saved-env) $\sigma$)))
---------------------------------------------------------------------

(value-of (call-exp fun $a_1$) env $\sigma$) = ($val_1$, $\sigma_1$)
(value-of (call-exp (proc->expval $val_1$) $a_2$) env $\sigma_1$) = ($val_2$, $\sigma_2$)
...
(value-of (call-exp (proc->expval $val_{n-1}$) $a_n$) env $\sigma_{n-1}$) = ($val_n$, $\sigma_n$)

\end{lstlisting}

\end{document}

