\documentclass[11pt]{article}

\usepackage{sectsty}
\usepackage{graphicx}
\usepackage{listings}
\usepackage{amsmath}
\usepackage[normalem]{ulem}
\usepackage[dvipsnames]{xcolor}
\usepackage{float}

% Colors for Non-Terminals in Grammar
\newcommand\Program{\textcolor{Orange}{\textbf{Program}}}
\newcommand\Expression{\textcolor{Red}{\textbf{Expression}}}
\newcommand\Number{\textcolor{ForestGreen}{\textbf{Number}}}
\newcommand\Identifier{\textcolor{BlueViolet}{\textbf{Identifier}}}
\newcommand\Digit{\textcolor{Cerulean}{\textbf{Digit}}}
\newcommand\Alphab{\textcolor{Tan}{\textbf{Alphabetic}}}


% Margins
\topmargin=-0.45in
\evensidemargin=0in
\oddsidemargin=0in
\textwidth=6.5in
\textheight=9.0in
\headsep=0.25in

\title{ Tarea 07 - El Lenguaje LETREC}
\author{ Enrique Giottonini, Miguel Navarro }
\date{Noviembre 09, 2022}

\begin{document}
\maketitle	

\section*{Especificación del Lenguaje.}
\subsection*{Sintáxis Concreta}
\begin{table}[h]
\begin{tabular}{lll}
\Program    & :=  & \Expression \\ \\
\Expression & :=  & \Number                                                           	\\
            & $|$ & -(\Expression, \Expression)                                   		\\
            & $|$ & zero? (\Expression)                                          	 	\\
            & $|$ & if \Expression \, then \Expression \, else \Expression        		\\
            & $|$ & \Identifier                                                  		\\
            & $|$ & let \Identifier \, = \Expression \,  in \Expression           		\\
            & $|$ & proc (\Identifier) \Expression                                		\\
            & $|$ & (\Expression \, \Expression)                                  		\\
            & $|$ & letrec \Identifier(\Identifier) = \Expression \, in \Expression  \\ \\
\Digit      & :=  & 0 $|$  1 $|$ 2 $|$ 3 $|$ 4 $|$ 5 $|$ 6 $|$ 7 $|$ 8 $|$ 9      	\\ \\
\Number     & :=  & \Digit                       		\\
            & $|$ & \Digit \Number         				\\
            & $|$ & \Digit \,.\Digit$\{\Digit\}^*$   \\ \\
\Alphab     &  =  & [a-zA-Z]                         \\ \\
\Identifier & :=  & \Alphab\Identifier                  \\
            & $|$ & \Identifier$\{\Digit\}^*$           \\
                                                       
\end{tabular}
\end{table}
\newpage
\subsection*{Sintáxis Abstracta (Notación de Racket)}
\Program: \\
- (a-program exp1) \\

\noindent \Expression: \\
- (const-exp num) \\
- (diff-exp exp1 exp2) \\
- (zero?-exp exp1) \\
- (if-exp exp1 exp2 exp3) \\
- (var-exp var) \\
- (let-exp var exp1 body) \\
- (proc-exp var body) \\ 
- (call-exp rator rand) \\
- (letrec-exp p-name b-var p-body letrec-body) \\

\noindent \Number: \verb|Real|

\noindent \Identifier: Versión limítada de \verb|Symbol|

\subsection*{Semántica}

\begin{lstlisting}[mathescape]
(value-of (const-exp n) $\rho$) = (num-val n)

(value-of (var-exp var) $\rho$) = (apply-$\rho$ $\rho$ var)

(value-of (diff-exp exp1 exp2) $\rho$)
 = (num-val
    (- (expval->num (value-of exp1 $\rho$))
       (expval->num (value-of exp2 $\rho$))))

(value-of (zero?-exp exp1) $\rho$)
 = (if (equal? 0 (expval->num (value-of exp1 $\rho$)))
       (bool-val #t)
       (bool-val #f))

(value-of (if-exp exp1 exp2 exp3) $\rho$)
 = (if (expval->bool (value-of exp1 $\rho$))
       (value-of exp2 $\rho$)
       (value-of exp3 $\rho$))

(value-of (let-exp var exp1 body) $\rho$)
 = (value-of body (extend-env var (value-of exp1 $\rho$) $\rho$))

(value-of (proc-exp var body) $\rho$)
 = (proc-val (procedure var body $\rho$))

(value-of (call-exp rator rand) $\rho$)
 = (let ([proc (expval->proc (value-of rator $\rho$))]
         [arg (value-of rand $\rho$)])
     (apply-procedure proc arg))
     
(value-of (letrec-exp proc-name bound-var proc-body letrec-body) $\rho$)
 = (value-of letrec-body (extend-env-rec proc-name bound-var proc-body $\rho$))




\end{lstlisting}

\end{document}

